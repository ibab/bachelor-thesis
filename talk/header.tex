\usepackage{fixltx2e}
\usepackage{letltxmacro}
\usepackage{expl3}
\usepackage{xparse}
\usepackage{luacode}
\usepackage{luatexbase}
\RequireLuaModule{lualibs}
\usepackage{metalogo}

\usepackage{xcolor}
\usepackage{luacolor}

\usepackage[ngerman]{babel}
\usepackage[ngerman]{translator}
\usepackage[backend=biber,sortlocale=de_DE.UTF-8]{biblatex}
\usepackage[notbib,nottoc]{tocbibind}

\usepackage[sumlimits,intlimits,namelimits]{amsmath}
\usepackage{amssymb}
\usepackage{upref}
\usepackage{mathtools}
\usepackage[no-math]{fontspec} % after amssymb
\usepackage[italic,noss]{hepnicenames} % loads bm, before unicode-math
\usepackage[math-style=ISO,bold-style=ISO,sans-style=italic,nabla=upright,partial=upright,vargreek-shape=unicode,warnings-off={mathtools-colon}]{unicode-math}
\usepackage[retainorgcmds]{IEEEtrantools}
\usepackage{tensor}
\usepackage[version=3]{mhchem}
\usepackage{siunitx}
\usepackage{esvect}

\usepackage{pdflscape}
\usepackage{float}
\usepackage[above,below,section]{placeins}
\usepackage{flafter}
\usepackage[margin=10pt,font=small,labelfont=bf]{caption}
\usepackage{subcaption}
%\usepackage{subfig}
\usepackage{graphicx}
\usepackage{array}
\usepackage{multirow}
\usepackage{booktabs}

\usepackage[stable,bottom,hang]{footmisc}
\usepackage[strict]{csquotes}
\usepackage{hyphenat}
\usepackage{textcmds}
\usepackage{xspace}
\usepackage{eurosym}
%\usepackage{minted}

% Tikz %%%%%%%%%

\usepackage{tikz}
\usetikzlibrary{calc}
\usetikzlibrary{positioning}
\usetikzlibrary{shapes.geometric}
\usetikzlibrary{circuits.logic.IEC,circuits.ee.IEC}
\usetikzlibrary{arrows,shapes}
\usetikzlibrary{trees}
\usetikzlibrary{calc,through}
\usetikzlibrary{decorations.pathmorphing}	% For Feynman Diagrams
\usetikzlibrary{decorations.markings}

\pgfdeclaredecoration{complete sines}{initial}
{
    \state{initial}[
        width=+0pt,
        next state=upsine,
        persistent precomputation={\pgfmathsetmacro\matchinglength{
            \pgfdecoratedinputsegmentlength / int(\pgfdecoratedinputsegmentlength/\pgfdecorationsegmentlength)}
            \setlength{\pgfdecorationsegmentlength}{\matchinglength pt}
        }] {}
    \state{upsine}[width=\pgfdecorationsegmentlength,next state=downsine]{
        \pgfpathsine{\pgfpoint{0.50\pgfdecorationsegmentlength}{0.5\pgfdecorationsegmentamplitude}}
        \pgfpathcosine{\pgfpoint{0.50\pgfdecorationsegmentlength}{-0.5\pgfdecorationsegmentamplitude}}
    }
    \state{downsine}[width=\pgfdecorationsegmentlength,next state=upsine]{
        \pgfpathsine{\pgfpoint{0.50\pgfdecorationsegmentlength}{-0.5\pgfdecorationsegmentamplitude}}
        \pgfpathcosine{\pgfpoint{0.50\pgfdecorationsegmentlength}{0.5\pgfdecorationsegmentamplitude}}
}
    \state{final}{}
}

\tikzset{
	% >=stealth', %%  Uncomment for more conventional arrows
    vector/.style={decorate, decoration={complete sines,segment length=6}, draw},
	  provector/.style={decorate, decoration={snake,amplitude=2.5pt}, draw},
	  antivector/.style={decorate, decoration={snake,amplitude=-2.5pt}, draw},
    fermion/.style={draw=black, postaction={decorate},
        decoration={markings,mark=at position .55 with {\arrow[draw=black]{>}}}},
    fermionbar/.style={draw=black, postaction={decorate},
        decoration={markings,mark=at position .55 with {\arrow[draw=black]{<}}}},
    fermionnoarrow/.style={draw=black},
    gluon/.style={decorate, draw=black,
        decoration={coil,amplitude=4pt, segment length=5pt}},
    scalar/.style={dashed,draw=black, postaction={decorate},
        decoration={markings,mark=at position .55 with {\arrow[draw=black]{>}}}},
    scalarbar/.style={dashed,draw=black, postaction={decorate},
        decoration={markings,mark=at position .55 with {\arrow[draw=black]{<}}}},
    scalarnoarrow/.style={dashed,draw=black},
    electron/.style={draw=black, postaction={decorate},
        decoration={markings,mark=at position .55 with {\arrow[draw=black]{>}}}},
	bigvector/.style={decorate, decoration={snake,amplitude=4pt}, draw},
}
\tikzstyle{block} = [draw, rectangle, 
    minimum height=3em, minimum width=6em]


%%%%%%%%%%%%%%%%

\usepackage[unicode=true,pdfcreator={},pdfproducer={}]{hyperref}
%\usepackage{bookmark}
\usepackage[shortcuts]{extdash} % must be after hyperref for shortcuts
\usepackage{listings}


\input{modules/base.tex}
\input{modules/comment.tex}
\input{modules/floats.tex}
\input{modules/math.tex}
\input{modules/tikz.tex}
\input{modules/units.tex}

\ExplSyntaxOn

%patch maybe for hepnames
\RenewDocumentCommand \maybebm {m}
{
  \ensuremath{{{#1}}}
}
\RenewDocumentCommand \maybesf {m}
{
  \ensuremath{{#1}}
}
\RenewDocumentCommand \mayberm {m}
{
  \ensuremath{{\mathrm{#1}}}
}
\RenewDocumentCommand \maybeitrm {m}
{
  \ensuremath{{\mathrm{#1}}}
}
\RenewDocumentCommand \maybeitsubscript {m}
{
  \ensuremath{{#1}}
}

\setmainfont[BoldFont = {LMRoman10-Bold}, % not optimal
             ItalicFont = {LMRoman10-Italic}, % not optimal
             BoldItalicFont = {LMRoman10-BoldItalic}, % not optimal
             SmallCapsFont = {Latin Modern Roman Caps},
             SlantedFont = {Latin Modern Roman Slanted},
             ItalicFeatures  = {
               SmallCapsFont = {LMRomanCaps10-Oblique}
             },
            ]{Latin Modern Roman}

\setsansfont{Latin Modern Sans}

\setmonofont[SmallCapsFont = {Latin Modern Mono Caps},
             SlantedFont = {Latin Modern Mono Slanted},
             ItalicFeatures  = {
               SmallCapsFont = {LMMonoCaps10-Oblique}
             },
            ]{Latin Modern Mono}

\RenewDocumentCommand\hangfootparskip{}{0pt}
\RenewDocumentCommand\hangfootparindent{}{15pt}

\mathtoolsset{mathic}

\setmathfont{Latin Modern Math}

\setmathfont[range={\mathscr,\mathbfscr}]{XITS Math}
%\setmathfont[range={\mathcal,\mathbfcal},StylisticSet=1]{XITS Math}
\setmathfont[range={\mathcal,\mathbfcal}]{Latin Modern Math}

% much nicer than latin Modern or XITS
\DeclareSymbolFont{AMSb}{U}{msb}{m}{n}
\protected\def\mathbb#1{{\mathchar\numexpr256*\symAMSb+`#1\relax}}

% define fallbacks here
\setmathfont[range=\vDash]{XITS Math}
\setmathfont[range=\coloneq]{XITS Math}
\setmathfont[range=\propto]{XITS Math}
\setmathfont[range={\lblkbrbrak,\rblkbrbrak}]{XITS Math}

% make bar horizontal, use \hslash for slashed h
\let\hbar\relax
\DeclareMathSymbol{\hbar}{\mathord}{AMSb}{"7E}

\NewDocumentCommand \mathdefault {}
{
  \mathit
}

%\RenewDocumentCommand \header_math_tensor_font {}
%{
%  \mathit
%}

%\removenolimits{\int}
%\removenolimits{\iint}
%\removenolimits{\iiint}
%\removenolimits{\iiiint}
%\removenolimits{\oint}
%\removenolimits{\oiint}
%\removenolimits{\oiiint}
%\removenolimits{\intclockwise}
%\removenolimits{\varointclockwise}
%\removenolimits{\ointctrclockwise}
%\removenolimits{\sumint}
%\removenolimits{\intbar}
%\removenolimits{\intBar}
%\removenolimits{\fint}
%\removenolimits{\cirfnint}
%\removenolimits{\awint}
%\removenolimits{\rppolint}
%\removenolimits{\scpolint}
%\removenolimits{\npolint}
%\removenolimits{\pointint}
%\removenolimits{\sqint}
%\removenolimits{\intlarhk}
%\removenolimits{\intx}
%\removenolimits{\intcap}
%\removenolimits{\intcup}
%\removenolimits{\upint}
%\removenolimits{\lowint}

% make delimiters always grow, never have two levels the same size
\setlength{\delimitershortfall}{-1sp}

% add more space between equations, default is 3pt
\setlength{\IEEEnormaljot}{10pt}

\sisetup
{
  strict,
  locale=US,
  per-mode=symbol-or-fraction,
  separate-uncertainty=true,
  add-integer-zero=true,
  add-decimal-zero=true,
  round-integer-to-decimal=true,
  table-align-exponent=true,
  table-align-uncertainty=true,
  table-unit-alignment=center,
  math-ohm=\mathup{\Omega},
  text-ohm=\ensuremath{\mathup{\Omega}}
}

\hypersetup{
  unicode=true,
  pdfcreator={},
  pdfproducer={},
}

\addbibresource{main.bib}

\nocite{numpy, scipy, matplotlib, uncertainties, root, roofit}

\NewDocumentCommand \makebibliography {}
{
  \printbibliography[heading=bibintoc]
}

\ExplSyntaxOff
