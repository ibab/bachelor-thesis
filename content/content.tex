
\section{Einleitung}

\section{Standardmodell der Teilchenphysik}

\subsection{Flavor-Sektor im Standardmodell}

% 6 Quark-Flavors
% Ladung: 2/3, -1/3
% Entdeckt: Deep Inelastic Scattering (SLAC)
% starke Wechselwirkung
% Elektromagnetismus
% schwache Wechselwirkung: CKM-Matrix und Quark-Mixing

\subsection{CP-Verletzung im Standardmodell}

% Was ist CP?
% Komplexe Phase in der CKM-Matrix -> CP-Verletzung

\subsection{$\PB$-$\PaB$-Oszillation}

\begin{figure}
  \begin{center}
  \begin{tikzpicture}[line width=1.2 pt, scale=1.5]
    \draw[fermion] (-1,2.3) -- node [above] {$d$} (0,2);
    \draw[fermion] (0,2) -- node [above] {$u,s,t$} (2, 2);
    \draw[fermion] (2,2) -- node [above] {$b$} (3,2.3);
     
    \draw[fermionbar] (-1,-.3) -- node [above] {$\bar{b}$} (0,0);
    \draw[fermionbar] (0,0) -- node [above] {$\bar{u},\bar{s},\bar{t}$} (2,0);
    \draw[fermionbar] (2,0) -- node [above] {$\bar{d}$} (3,-.3);
    
    \draw[vector] (0,2) -- node [left] {$W^\pm$} (0,0);
    \draw[vector] (2,2) -- node [right] {$W^\mp$} (2,0);
  \end{tikzpicture}
  \end{center}

  \label{B_oscillation}
  \caption{bla}
\end{figure}

\section{Experimentelle Grundlagen}

\subsection{Der LHCb-Detektor}

\subsection{Flavor-Tagging}

% Warum Flavor-Tagging? -> Man will an Produktions-Flavor des b kommen
% Keine triviale Lösung
% Interessant: Kanäle mit self-tagging final state
%  z.B. B0 -> Dpi, B0 -> JpsiKst
% Definition OST und SST (pion, kaon)
% Mistag-Wahrscheinlichkeit ω, Tagging-Dilution D = 1 - 2ω
% Tagger geben Mistag-Schätzung η heraus, muss kalibriert werden, dazu Kanäle mit self-tagging final state
% ε_eff, ε_tag

\subsection{Das SPlot-Verfahren?}

\section{Datensatz}

\section{\texorpdfstring{Kalibration von $\PBzero \to \PJpsi \PKst$}{Kalibration von B0 -> JpsiKst}}

% Der Plan:
% Datensatz je nach η in 5 Kategorien aufteilen (Strategie überlegen, noch selbst implementieren?)
% Simultanfit des geteilten Datensatzes (yields und ωs individuell, vielleicht auch Akzeptanzparameter)
% Auf dem gesamten Datensatz Background-Trennung per SPlot und Berechnung des mittleren η in jeder Kategorie
% Anschließend linearer (quadratischer) Fit der 5 η-ω-Punkte, Parameter p1, p2, (p3) ermitteln
% Entscheidung: Was passt besser? (quadratisch auf SSπ, sonst linear)
% Erzeugung eines neuen, kalibrierten Datensatzes
% Wiederholung der Prozedur: Jetzt sollte ω ≈ η rauskommen

\subsection{Parametrisierung der Likelihood-Funktion}

% m-t Fit in 5 Kategorien:
% Signal-pdf und zwei kombinatorische backgrounds
% jeweils Massen-pdf
% jeweils Zeit-pdf
% wie genau mit RooFit implementiert?

% m-Fit für SPlot
% es werden die Massen-pdfs von oben verwendet

\subsection{Fitresultate}

% Alle Parameter (5 Kategorien) als Tabelle (Fehler mit Minos)
% Alle Parameter (Massen-Fit) als Tabelle (Fehler mit Minos)
% nach Background-Trennung mit SPlot: Mittelwerte von η bestimmt

\subsection{Ermittlung der Kalibrationsparameter}

% Tabelle der ermittelten ω, η
% Plot mit linearem Fit
% Im Text nochmal p1, p2, (p3) auflisten 

\section{Schlussfolgerungen}

% vim: set ft=tex:
