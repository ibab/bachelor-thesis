
\section{Einleitung}

\section{Standardmodell der Teilchenphysik}

\subsection{Flavor-Sektor im Standardmodell}

\subsection{CP-Verletzung im Standardmodell}

\subsection{$\PB$-Oszillation}

\section{Experimentelle Grundlagen}

\subsection{Aufbau des LHCb-Detektors?}

\subsection{Flavor-Tagging}

% Warum Flavor-Tagging? -> Man will an Produktions-Flavor des b kommen
% Keine triviale Lösung
% Interessant: Kanäle mit self-tagging final state
%  z.B. B0 -> Dpi, B0 -> JpsiKst
% Definition OST und SST (pion, kaon)
% Mistag-Wahrscheinlichkeit ω, Tagging-Dilution D = 1 - 2ω
% Tagger geben Mistag-Schätzung η heraus, muss kalibriert werden, dazu Kanäle mit self-tagging final state
% ε_eff, ε_tag

\subsection{Das SPlot-Verfahren?}

\section{Datensatz}

\section{\texorpdfstring{Kalibration von $\PBzero \to \PJpsi \PKst$}{Kalibration von B0 -> JpsiKst}}

% Der Plan:
% Datensatz je nach η in 5 Kategorien aufteilen (Strategie überlegen, noch selbst implementieren?)
% Simultanfit des geteilten Datensatzes (yields und ωs individuell, vielleicht auch Akzeptanzparameter)
% Auf dem gesamten Datensatz Background-Trennung per SPlot und Berechnung des mittleren η in jeder Kategorie
% Anschließend linearer (quadratischer) Fit der 5 η-ω-Punkte, Parameter p1, p2, (p3) ermitteln
% Entscheidung: Was passt besser? (quadratisch auf SSπ, sonst linear)
% Erzeugung eines neuen, kalibrierten Datensatzes
% Wiederholung der Prozedur: Jetzt sollte ω ≈ η rauskommen

\subsection{Parametrisierung der Likelihood-Funktion}

% m-t Fit in 5 Kategorien:
% Signal-pdf und zwei kombinatorische backgrounds
% jeweils Massen-pdf
% jeweils Zeit-pdf
% wie genau mit RooFit implementiert?

% m-Fit für SPlot
% es werden die Massen-pdfs von oben verwendet

\subsection{Fitresultate}

% Alle Parameter (5 Kategorien) als Tabelle (Fehler mit Minos)
% Alle Parameter (Massen-Fit) als Tabelle (Fehler mit Minos)
% nach Background-Trennung mit SPlot: Mittelwerte von η bestimmt

\subsection{Ermittlung der Kalibrationsparameter}

% Tabelle der ermittelten ω, η
% Plot mit linearem Fit
% Im Text nochmal p1, p2, (p3) auflisten 

\section{Schlussfolgerungen}

% vim: set ft=tex:
