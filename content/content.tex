
\section{Einleitung}

\section{Standardmodell der Teilchenphysik}

Das Standardmodell der Teilchenphysik ist ein Quantenfeldtheorie, die alle bislang bekannten Elementarteilchen und ihre Interaktion über die starke, schwache und elektromagnetische Wechselwirkung beschreibt.
Es basiert auf der Symmetrie
\begin{eqn}
  SU(3)_c \prod SU(2)_L \prod U(1)_Y\:,
\end{eqn}
wobei $SU(3)_c$ die Eichsymmetrie der starken Wechselwirkung und $SU(2)_L \prod U(1)_Y$ die der elektroschwachen Wechselwirkung beschreibt.

Trotz der Erfolge des Standardmodells kann es sich dabei nicht um eine endgültige Theorie aller physikalischer Effekte handeln.
So gibt es noch keine allgemein akzeptierte Methode, um Gravitation in das Standardmodell mit einzubeziehen.
Außerdem ist eine Reihe von Phänomenen bekannt, die sich nicht durch das Standardmodell erklären lassen, wie zum Beispiel das Vorhandensein und die Natur von dunkler Materie und dunkler Energie, die unerklärt hohe Materie-Antimaterie-Asymmetrie im Universum, die Neutrinomassen, sowie das Verhalten des Standardmodells bei sehr hohen Energien (das Hierarchieproblem).

\subsection{Flavor-Sektor im Standardmodell}

% 6 Quark-Flavors

\begin{eqn}
  \begin{pmatrix}
    \Pqu & \Pqc & \Pqt \\
    \Pqd & \Pqs & \Pqb \\
  \end{pmatrix}
\end{eqn}

% Entdeckt: Deep Inelastic Scattering (SLAC)
% elektromagnetische Wechselwirkung: Ladung: 2/3, -1/3
% starke Wechselwirkung: Farbladung
% schwache Wechselwirkung: CKM-Matrix und Quark-Mixing

\begin{eqn}
  \begin{pmatrix}
    \Pqd' \\
    \Pqs' \\
    \Pqb' \\
  \end{pmatrix}
  =
  \begin{pmatrix}
    V_{\Pqu\Pqd} & V_{\Pqu\Pqs} & V_{\Pqu\Pqb} \\
    V_{\Pqc\Pqd} & V_{\Pqc\Pqs} & V_{\Pqc\Pqb} \\
    V_{\Pqt\Pqd} & V_{\Pqt\Pqs} & V_{\Pqt\Pqb} \\
  \end{pmatrix}
  \begin{pmatrix}
    \Pqd \\
    \Pqs \\
    \Pqb \\
  \end{pmatrix}
\end{eqn}

\newcommand{\Vud}{V_{\Pqu\Pqs}}
\newcommand{\Vus}{V_{\Pqu\Pqs}}
\newcommand{\Vub}{V_{\Pqu\Pqb}}
\newcommand{\Vcd}{V_{\Pqc\Pqd}}
\newcommand{\Vcs}{V_{\Pqc\Pqs}}
\newcommand{\Vcb}{V_{\Pqc\Pqb}}
\newcommand{\Vtd}{V_{\Pqt\Pqd}}
\newcommand{\Vts}{V_{\Pqt\Pqs}}
\newcommand{\Vtb}{V_{\Pqt\Pqb}}

% bestes dreieck
\begin{eqn}
  \Vud\Vub^* + \Vcd\Vcb^* + \Vtd\Vtb^* = 0
\end{eqn}

% oben links
\begin{eqn}
  \abs{\frac{\Vud \Vub^*}{\Vcd \Vcb^*}}
\end{eqn}

% oben rechts
\begin{eqn}
  \abs{\frac{\Vtd \Vtb^*}{\Vcd \Vcb^*}}
\end{eqn}

\begin{center}
  \begin{tikzpicture}[scale=10]
    \coordinate (A) at (0.25,0.70);
    \coordinate (B) at (1,0);
    \coordinate (C) at (0,0);
    \draw (A)--(B)--(C)--cycle;
    
    \tkzLabelSegment[above left=0.15](A,C){$\abs{\frac{\Vud \Vub^*}{\Vcd \Vcb^*}}$}
    \tkzLabelSegment[above right=0.15](B,A){$\abs{\frac{\Vtd \Vtb^*}{\Vcd \Vcb^*}}$}
    
    \tkzMarkAngle[size=0.35](C,A,B)
    \tkzLabelAngle[pos = 0.2](C,A,B){$\alpha$}
    
    \tkzMarkAngle[size=0.35](A,B,C)
    \tkzLabelAngle[pos = 0.2](A,B,C){$\beta$}

    \tkzMarkAngle[size=0.35](B,C,A)
    \tkzLabelAngle[pos = 0.2](B,C,A){$\gamma$}
  \end{tikzpicture}
\end{center}

% 2008: Nobelpreis für Kobayashi und Maskawa
% Unitaritätsdreiecke
% interessant: CP-Verletzung

\subsection{CP-Verletzung im Standardmodell}

\cite{wu}
\cite{cp-lee-yang}
\cite{kaons-cronin-fitch}

% Was ist CP?
% Momentane Menge reicht nicht, um Materie-Antimaterie-Asymmetrie zu erklären
% Komplexe Phase in der CKM-Matrix -> CP-Verletzung
% CP-Dreieck
% sin(2β)

\subsection{$\PB$-$\PaB$-Oszillation}

\begin{figure}
  \begin{tikzpicture}[line width=1.2 pt, scale=1.1]
    \draw[fermion] (-1,2.3) node [left] {\Pqd} -- (0,2);
    \draw[fermion] (0,2) -- node [above] {$\Pqu,\Pqc,\Pqt$} (2, 2);
    \draw[fermion] (2,2) -- (3,2.3) node [right] {\Pqb};
     
    \draw[fermionbar] (-1,-.3) node [left] {\Pqb} -- (0,0);
    \draw[fermionbar] (0,0) -- node [above] {$\Pqu,\Pqc,\Pqt$} (2,0);
    \draw[fermionbar] (2,0) -- (3,-.3) node [right] {\Pqd};
    
    \draw[vector] (0,2) -- node [left] {\PW} (0,0);
    \draw[vector] (2,2) -- node [right] {\PW} (2,0);
  \end{tikzpicture}
  \hspace{1cm}
  \begin{tikzpicture}[line width=1.2 pt, scale=1.1]
    \draw[fermion] (-1,2.3) node [left] {\Pqd} -- (0,2);
    \draw[vector] (0,2) -- node [above] {\PWm} (2, 2);
    \draw[fermion] (2,2) -- (3,2.3) node [right] {$b$};
     
    \draw[fermionbar] (-1,-.3) node [left] {\Pqb} -- (0,0);
    \draw[vector] (0,0) -- node [above] {\PWp} (2,0);
    \draw[fermionbar] (2,0) -- (3,-.3) node [right] {\Pqd};
    
    \draw[fermion] (0,2) -- node [left] {$\Pqu,\Pqc,\Pqt$} (0,0);
    \draw[fermionbar] (2,2) -- node [right] {$\Pqu,\Pqc,\Pqt$} (2,0);
  \end{tikzpicture}

  \label{B_oscillation}
  \caption{bla}
\end{figure}

%\begin{tikzpicture}[line width=1.2 pt, scale=1]
%  \draw[fermion] (-1,2.3) -- (0,2);
%  \draw[vector] (0,2) arc (160:0:1.025);
%  \draw[fermion] (2,2) -- (3,2.3);
%  %
%  \draw[fermion] (0,2) -- (1,1);
%  \draw[fermion] (1,1) -- (2,2);
%  %
%  \draw[vector] (1,1) -- (1,-.5);
%  \draw[fermion] (3,-1) -- (1,-.5);
%  \draw[fermion] (1,-.5) -- (-1,-1);
%\end{tikzpicture}

\section{Experimentelle Grundlagen}

\subsection{Der LHCb-Detektor}

\subsection{Flavor-Tagging}

% Warum Flavor-Tagging? -> Man will an Produktions-Flavor des b kommen
% Keine triviale Lösung
% Interessant: Kanäle mit self-tagging final state
%  z.B. B0 -> Dpi, B0 -> JpsiKst
% Definition OST und SST (pion, kaon)
% Mistag-Wahrscheinlichkeit ω, Tagging-Dilution D = 1 - 2ω
% Tagger geben Mistag-Schätzung η heraus, muss kalibriert werden, dazu Kanäle mit self-tagging final state
% ε_eff, ε_tag

\subsection{Das SPlot-Verfahren?}

\section{Datensatz}

\section{\texorpdfstring{Kalibration von $\PBzero \to \PJpsi \PKst$}{Kalibration von B0 -> JpsiKst}}

% Der Plan:
% Datensatz je nach η in 5 Kategorien aufteilen (Strategie überlegen, noch selbst implementieren?)
% Simultanfit des geteilten Datensatzes (yields und ωs individuell, vielleicht auch Akzeptanzparameter)
% Auf dem gesamten Datensatz Background-Trennung per SPlot und Berechnung des mittleren η in jeder Kategorie
% Anschließend linearer (quadratischer) Fit der 5 η-ω-Punkte, Parameter p1, p2, (p3) ermitteln
% Entscheidung: Was passt besser? (quadratisch auf SSπ, sonst linear)
% Erzeugung eines neuen, kalibrierten Datensatzes
% Wiederholung der Prozedur: Jetzt sollte ω ≈ η rauskommen

\subsection{Parametrisierung der Likelihood-Funktion}

% m-t Fit in 5 Kategorien:
% Signal-pdf und zwei kombinatorische backgrounds
% jeweils Massen-pdf
% jeweils Zeit-pdf
% wie genau mit RooFit implementiert?

% m-Fit für SPlot
% es werden die Massen-pdfs von oben verwendet

\subsection{Fitresultate}

% Alle Parameter (5 Kategorien) als Tabelle (Fehler mit Minos)
% Alle Parameter (Massen-Fit) als Tabelle (Fehler mit Minos)
% nach Background-Trennung mit SPlot: Mittelwerte von η bestimmt

\subsection{Ermittlung der Kalibrationsparameter}

% Tabelle der ermittelten ω, η
% Plot mit linearem Fit
% Im Text nochmal p1, p2, (p3) auflisten 

\section{Schlussfolgerungen}

% vim: set ft=tex:
