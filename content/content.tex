
\section{Einleitung}

\section{Standardmodell der Teilchenphysik}

Das Standardmodell der Teilchenphysik ist ein Quantenfeldtheorie, die alle bislang bekannten Elementarteilchen und ihre Interaktion über die starke, schwache und elektromagnetische Wechselwirkung beschreibt.
Es basiert auf der Symmetrie
\begin{eqn}
  SU(3)_c \otimes SU(2)_L \otimes U(1)_Y\:,
\end{eqn}
wobei $SU(3)_c$ die Eichsymmetrie der starken Wechselwirkung und $SU(2)_L \otimes U(1)_Y$ die der elektroschwachen Wechselwirkung beschreibt.

Trotz der Erfolge des Standardmodells kann es sich dabei nicht um eine endgültige Theorie aller physikalischer Effekte handeln.
So gibt es noch keine allgemein akzeptierte Methode, um Gravitation in das Standardmodell mit einzubeziehen.
Außerdem ist eine Reihe von Phänomenen bekannt, die sich nicht durch das Standardmodell erklären lassen, wie zum Beispiel das Vorhandensein und die Natur von dunkler Materie und dunkler Energie, die unerklärt hohe Materie-Antimaterie-Asymmetrie im Universum, die Neutrinomassen, sowie das Verhalten des Standardmodells bei sehr hohen Energien (das Hierarchieproblem).

\subsection{Flavor-Sektor im Standardmodell}

% 6 Quark-Flavors

\begin{eqn}
  \begin{pmatrix}
    \Pqu & \Pqc & \Pqt \\
    \Pqd & \Pqs & \Pqb \\
  \end{pmatrix}
\end{eqn}

% Entdeckt: Deep Inelastic Scattering (SLAC)
% elektromagnetische Wechselwirkung: Ladung: 2/3, -1/3
% starke Wechselwirkung: Farbladung
% schwache Wechselwirkung: CKM-Matrix und Quark-Mixing

\begin{eqn}
  \begin{pmatrix}
    \Pqd' \\
    \Pqs' \\
    \Pqb' \\
  \end{pmatrix}
  =
  \begin{pmatrix}
    V_{\Pqu\Pqd} & V_{\Pqu\Pqs} & V_{\Pqu\Pqb} \\
    V_{\Pqc\Pqd} & V_{\Pqc\Pqs} & V_{\Pqc\Pqb} \\
    V_{\Pqt\Pqd} & V_{\Pqt\Pqs} & V_{\Pqt\Pqb} \\
  \end{pmatrix}
  \begin{pmatrix}
    \Pqd \\
    \Pqs \\
    \Pqb \\
  \end{pmatrix}
\end{eqn}

\newcommand{\Vud}{V_{\Pqu\Pqs}}
\newcommand{\Vus}{V_{\Pqu\Pqs}}
\newcommand{\Vub}{V_{\Pqu\Pqb}}
\newcommand{\Vcd}{V_{\Pqc\Pqd}}
\newcommand{\Vcs}{V_{\Pqc\Pqs}}
\newcommand{\Vcb}{V_{\Pqc\Pqb}}
\newcommand{\Vtd}{V_{\Pqt\Pqd}}
\newcommand{\Vts}{V_{\Pqt\Pqs}}
\newcommand{\Vtb}{V_{\Pqt\Pqb}}

% bestes dreieck
\begin{eqn}
  \Vud\Vub^* + \Vcd\Vcb^* + \Vtd\Vtb^* = 0
\end{eqn}

% oben links
\begin{eqn}
  \abs{\frac{\Vud \Vub^*}{\Vcd \Vcb^*}}
\end{eqn}

% oben rechts
\begin{eqn}
  \abs{\frac{\Vtd \Vtb^*}{\Vcd \Vcb^*}}
\end{eqn}

\begin{center}
  \begin{tikzpicture}[scale=6]
    \coordinate (A) at (0.25,0.70);
    \coordinate (B) at (1,0);
    \coordinate (C) at (0,0);
    \draw (A)--(B)--(C)--cycle;
    
    \tkzLabelSegment[above left=0.15](A,C){$\abs{\frac{\Vud \Vub^*}{\Vcd \Vcb^*}}$}
    \tkzLabelSegment[above right=0.15](B,A){$\abs{\frac{\Vtd \Vtb^*}{\Vcd \Vcb^*}}$}
    
    \tkzMarkAngle[size=0.35](C,A,B)
    \tkzLabelAngle[pos = 0.2](C,A,B){$\alpha$}
    
    \tkzMarkAngle[size=0.35](A,B,C)
    \tkzLabelAngle[pos = 0.2](A,B,C){$\beta$}

    \tkzMarkAngle[size=0.35](B,C,A)
    \tkzLabelAngle[pos = 0.2](B,C,A){$\gamma$}
  \end{tikzpicture}
\end{center}

% 2008: Nobelpreis für Kobayashi und Maskawa
% Unitaritätsdreiecke
% interessant: CP-Verletzung

\subsection{CP-Verletzung im Standardmodell}

% Entdeckung
\cite{wu}
\cite{cp-lee-yang}

% Kaon system
\cite{kaons-cronin-fitch}

% Was ist CP?
% Momentane Menge reicht nicht, um Materie-Antimaterie-Asymmetrie zu erklären
% Komplexe Phase in der CKM-Matrix -> CP-Verletzung
% CP-Dreieck
% sin(2β) -> ermittelt über Amplitude der Mixing-Funktion
% CP-Verletzung senkt Amplitude ab, ebenso wie andere Effekte: Akzeptanz, Mistag-Wahrscheinlichkeit
% daher: Extrem wichtig die Effekte zu bestimmen

\subsection{$\PB$-$\PaB$-Oszillation}

% Warum interessant? -> sin(2β) lässt sich bestimmen, Test des Standardmodells

\begin{figure}
  \begin{tikzpicture}[line width=1.2 pt, scale=1.1]
    \draw[fermion] (-1,2.3) node [left] {\Pqd} -- (0,2);
    \draw[fermion] (0,2) -- node [above] {$\Pqu,\Pqc,\Pqt$} (2, 2);
    \draw[fermion] (2,2) -- (3,2.3) node [right] {\Pqb};
     
    \draw[fermionbar] (-1,-.3) node [left] {\Pqb} -- (0,0);
    \draw[fermionbar] (0,0) -- node [above] {$\Pqu,\Pqc,\Pqt$} (2,0);
    \draw[fermionbar] (2,0) -- (3,-.3) node [right] {\Pqd};
    
    \draw[vector] (0,2) -- node [left] {\PW} (0,0);
    \draw[vector] (2,2) -- node [right] {\PW} (2,0);
  \end{tikzpicture}
  \hspace{1cm}
  \begin{tikzpicture}[line width=1.2 pt, scale=1.1]
    \draw[fermion] (-1,2.3) node [left] {\Pqd} -- (0,2);
    \draw[vector] (0,2) -- node [above] {\PWm} (2, 2);
    \draw[fermion] (2,2) -- (3,2.3) node [right] {$b$};
     
    \draw[fermionbar] (-1,-.3) node [left] {\Pqb} -- (0,0);
    \draw[vector] (0,0) -- node [above] {\PWp} (2,0);
    \draw[fermionbar] (2,0) -- (3,-.3) node [right] {\Pqd};
    
    \draw[fermion] (0,2) -- node [left] {$\Pqu,\Pqc,\Pqt$} (0,0);
    \draw[fermionbar] (2,2) -- node [right] {$\Pqu,\Pqc,\Pqt$} (2,0);
  \end{tikzpicture}

  \label{B_oscillation}
  \caption{bla}
\end{figure}

%\begin{tikzpicture}[line width=1.2 pt, scale=1]
%  \draw[fermion] (-1,2.3) -- (0,2);
%  \draw[vector] (0,2) arc (160:0:1.025);
%  \draw[fermion] (2,2) -- (3,2.3);
%  %
%  \draw[fermion] (0,2) -- (1,1);
%  \draw[fermion] (1,1) -- (2,2);
%  %
%  \draw[vector] (1,1) -- (1,-.5);
%  \draw[fermion] (3,-1) -- (1,-.5);
%  \draw[fermion] (1,-.5) -- (-1,-1);
%\end{tikzpicture}

\section{Experimentelle Grundlagen}

% blabla

\subsection{Der LHCb-Detektor}

% Bild des Detektors
% 1, 2 Sätze zu den wichtigsten Komponenten

\subsection{Flavor-Tagging}

Zur Bestimmung der Oszillationfrequenz $\Delta m_{\Pqd}$ der produzierten $\PB$- und $\PaB$-Mesonen muss neben den Lebensdauern der einzelnen beobachteten Mesonen sowohl deren End-, als auch deren Anfangszustand bekannt sein.

Der Endzustand kann aus den Zerfallsprodukten des Mesons bestimmt werden.

Die Bestimmung des Anfangszustandes, die man als \emph{Flavor-Tagging} bezeichnet, erweist sich als deutlich schwieriger.
Das LHCb-Experiment verwendet hierzu neuronale Netze, die darauf trainiert werden unter Einbeziehung weiterer während des Events erfolgter Beobachtungen eine möglichst gute Abschätzung für den Typ des \PB-Mesons (den sogenannten \emph{Tag}) zu liefern. % DaVinci erwähnen

Diese Einschätzung ist nicht immer richtig; die Wahrscheinlichkeit, dass sie falsch ist bezeichnet man als Mistag-Wahrscheinlichkeit $\omega$.
Für die Bestimmung von $\Delta m_{\Pqd}$ ist es ausgesprochen wichtig neben der Tag-Entscheidung ($\PB$ oder $\PaB$) die Mistag-Wahrscheinlichkeit zu kennen, da sich diese als $D = (1 - 2\omega)$ in der Amplitude der Mixing-Wahrscheinlichkeit bemerkbar macht. Den Faktor $D$ bezeichnet man als \emph{Tagging-Dilution}.
% -> Zeigen
% Warum wichtig? Nicht genug Statistik um sie mitzubestimmen?

% OST, SST erklären (mit Bildern)

Interessant sind Zerfallskanäle, die es durch ihre Beschaffenheit ermöglichen, den Anfangszustand aus den Zerfallsprodukten zu bestimmen, die also über einen sogenannten self-tagging Endzustand verfügen.
Zu diesen Kanälen gehören $\PBz \to \PDp\Pgpm$ und $\PBz \to \PJpsi \PKs$.
% Interessant: Kanäle mit self-tagging final state
%  z.B. B0 -> Dpi, B0 -> JpsiKst

Die T

% Mistag-Wahrscheinlichkeit ω, Tagging-Dilution D = 1 - 2ω, Tagging power ε_eff = ε_tag D^2
% Tagger geben Mistag-Schätzung η heraus, muss kalibriert werden, dazu Kanäle mit self-tagging final state
% ε_eff, ε_tag

\subsection{Das SPlot-Verfahren?}

\section{Datensatz}

% Bd -> J/psi Kst 2012
% BDT

\section{\texorpdfstring{Kalibration des $SS\pi$-Taggers für $\PBzero \to \PJpsi \PKst$}{Kalibration des SSpi-Taggers für B0 -> JpsiKst}}

Zur Kalibration wird der Datensatz je nach Größe des Mistag-Wertes $\eta$ des neuronalen Netzes in fünf Kategorien aufgeteilt.
Das mittlere $η$ wird pro Kategorie berechnet.
Da hierbei möglichst nur Mistag-Werte des Signals verwendet werden sollen, wird vorher eine Trennung der Signalkomponente mittels des SPlot-Verfahrens durchgeführt.

Daraufhin wird ein simultaner Maximum-Likelihood-Fit mit Events aller Kategorien durchgeführt, wobei die Anzahl der Signal- und Untergrundereignisse sowie die Mistag-Wahrscheinlichkeit $\omega$ pro Kategorie einzeln gefittet werden.
Verwendet werden dabei die Massen- und Zerfallszeit-Verteilungen des Datensatzes, sowie die Kenntnis über die Kaon-Ladung, die eine Identifizierung des Anfangszustandes erlaubt.
Anschließend werden die 5 $(\eta, \omega)$-Wertepaare sowohl linear als auch quadratisch gefittet.

An dieser Stelle wird entschieden, welche der beiden Parametrisierungen für $\eta(\omega)$ (linear, quadratisch) vorzuziehen ist.
Mittels der Koeffizienten $p_i$ aus dem Fit lässt sich eine Kalibrierung des Datensatzes durchführen, indem die $η$-Werte des neuronalen Netzes in kalibrierte Mistag-Werte $η_C$ mittels der Parametrisierung umgerechnet werden.

Besonders interessant ist es, herauszufinden, ob die Kalibrations-Parametrisierung eines Kanals angewendet auf einen anderen Kanal wiederum eine sinnvolle Kalibration ergibt. Dies wäre ein Hinweis dafür, dass man die ermittelte Parametrisierung auch auf Kanälen verwenden könnte, die nicht über einen self-tagging Endzustand verfügen und daher auf eine korrekte Kenntnis der Mistag-Wahrscheinlichkeit angewiesen sind.

% Der Plan:
% Datensatz je nach η in 5 Kategorien aufteilen (Strategie überlegen, noch selbst implementieren?)
% Simultanfit des geteilten Datensatzes (yields und ωs individuell, vielleicht auch Akzeptanzparameter)
% Auf dem gesamten Datensatz Background-Trennung per SPlot und Berechnung des mittleren η in jeder Kategorie
% Anschließend linearer (quadratischer) Fit der 5 η-ω-Punkte, Parameter p1, p2, (p3) ermitteln
% Entscheidung: Was passt besser? (quadratisch auf SSπ, sonst linear)
% Erzeugung eines neuen, kalibrierten Datensatzes
% Wiederholung der Prozedur: Jetzt sollte ω ≈ η rauskommen

\subsection{Parametrisierung der Likelihood-Funktion}

% Signal - Masse
\begin{eqns}
  P_\t{sig}(m; \mu, \sigma_1, \sigma_2, \sigma_3) & = & f^{12}_{m;\t{sig}} G(m; \mu, \sigma_1) + \\
  && (1 - f^{12}_{m;\t{sig}}) f^{23}_{m;\t{sig}} G(m;\mu,\sigma_2) + \\
  &&(1 - (1 - f^{12}_{m;\t{sig}}) f^{23}_{m;\t{sig}}) G(m; \mu, \sigma_3)
\end{eqns}

% Signal - Zeit
\begin{eqns}
  P_\t{sig}(t,q;\tau,\Delta m_{\Pqd}, \omega, s_\t{sig} = M(t, q; \tau, \Delta m_{\Pqd}, \omega) \otimes R(t;s_\t{sig})
\end{eqns}

% m-t Fit in 5 Kategorien:
% Signal-pdf und zwei kombinatorische backgrounds
% jeweils Massen-pdf
% jeweils Zeit-pdf

% m-Fit für SPlot
% es werden die Massen-pdfs von oben verwendet

\begin{eqn}
  M_\t{sig}(t) = \frac{N_\t{unmixed}(t) - N_\t{mixed}(t)}{N_\t{unmixed}(t) + N_\t{mixed}(t)} = \cos(\Delta m_{\Pqd} t)
\end{eqn}

\url{git@github.com:ibab/thesis}

\subsection{Implementierung und Durchführung des Fits}

\subsubsection{Implementierung des Wahrscheinlichkeitsmodells mittels des \texttt{RooFit}-Frameworks}
% wie genau mit RooFit implementiert?
% Startwerte nachlesbar hier und da

% RooGaussian, RooExponential, RooDecay, RooBMixDecay, RooSimPdfBuilder

\subsubsection{Durchführung der Optimierung}

\subsection{Fitresultate}

\begin{figure}
  \includegraphics[width=\textwidth]{analysis/JpsiKst-SSp/plot.pdf}
  \caption{Resultate der linearen und quadratischen Parametrisierungen der tatsächlichen Mistag-Wahrscheinlichkeiten $\omega$ und der vom neuronalen Netz geschätzten Mistag-Wahrscheinlichkeiten $\eta$}
\end{figure}

% Alle Parameter (5 Kategorien) als Tabelle (Fehler mit Minos)
% Alle Parameter (Massen-Fit) als Tabelle (Fehler mit Minos)
% nach Background-Trennung mit SPlot: Mittelwerte von η bestimmt

\subsection{Ermittlung der Kalibrationsparameter}

% wie gefittet?
% Tabelle der ermittelten ω, η
% Plot mit linearem Fit
% Entkorrelierung: Nullpunkt wird zum Schwerpunkt verschoben
% Tabelle mit p1, p2, (p3) + Korrelationsmatrix

\section{Schlussfolgerungen}

% vim: set ft=tex:
