\newcommand{\difference}[1]{\mathrm{\Delta} #1}

\section{Einleitung}

\section{Standardmodell der Teilchenphysik}

Das Standardmodell der Teilchenphysik ist ein Quantenfeldtheorie, die alle bislang bekannten Elementarteilchen und ihre Interaktion über die starke, schwache und elektromagnetische Wechselwirkung beschreibt.
Es basiert auf der Symmetrie
\begin{eqn}
  SU(3)_c \otimes SU(2)_L \otimes U(1)_Y\:,
\end{eqn}
wobei $SU(3)_c$ die Eichsymmetrie der starken Wechselwirkung und $SU(2)_L \otimes U(1)_Y$ die der elektroschwachen Wechselwirkung beschreibt.

Trotz der Erfolge des Standardmodells kann es sich dabei nicht um eine endgültige Theorie aller physikalischer Effekte handeln.
So gibt es noch keine allgemein akzeptierte Methode, um Gravitation in das Standardmodell mit einzubeziehen.
Außerdem ist eine Reihe von Phänomenen bekannt, die sich nicht durch das Standardmodell erklären lassen, wie zum Beispiel das Vorhandensein und die Natur von dunkler Materie und dunkler Energie, die unerklärt hohe Materie-Antimaterie-Asymmetrie im Universum, die Neutrinomassen, sowie das Verhalten des Standardmodells bei sehr hohen Energien (das Hierarchieproblem).

\subsection{Flavor-Sektor im Standardmodell}

% 6 Quark-Flavors

\begin{eqn}
  \begin{pmatrix}
    \Pqu & \Pqc & \Pqt \\
    \Pqd & \Pqs & \Pqb \\
  \end{pmatrix}
\end{eqn}

% Entdeckt: Deep Inelastic Scattering (SLAC)
% elektromagnetische Wechselwirkung: Ladung: 2/3, -1/3
% starke Wechselwirkung: Farbladung
% schwache Wechselwirkung: CKM-Matrix und Quark-Mixing

\begin{eqn}
  \begin{pmatrix}
    \Pqd' \\
    \Pqs' \\
    \Pqb' \\
  \end{pmatrix}
  =
  \begin{pmatrix}
    V_{\Pqu\Pqd} & V_{\Pqu\Pqs} & V_{\Pqu\Pqb} \\
    V_{\Pqc\Pqd} & V_{\Pqc\Pqs} & V_{\Pqc\Pqb} \\
    V_{\Pqt\Pqd} & V_{\Pqt\Pqs} & V_{\Pqt\Pqb} \\
  \end{pmatrix}
  \begin{pmatrix}
    \Pqd \\
    \Pqs \\
    \Pqb \\
  \end{pmatrix}
\end{eqn}

\newcommand{\Vud}{V_{\Pqu\Pqs}}
\newcommand{\Vus}{V_{\Pqu\Pqs}}
\newcommand{\Vub}{V_{\Pqu\Pqb}}
\newcommand{\Vcd}{V_{\Pqc\Pqd}}
\newcommand{\Vcs}{V_{\Pqc\Pqs}}
\newcommand{\Vcb}{V_{\Pqc\Pqb}}
\newcommand{\Vtd}{V_{\Pqt\Pqd}}
\newcommand{\Vts}{V_{\Pqt\Pqs}}
\newcommand{\Vtb}{V_{\Pqt\Pqb}}

% bestes dreieck
\begin{eqn}
  \Vud\Vub^* + \Vcd\Vcb^* + \Vtd\Vtb^* = 0
\end{eqn}

% oben links
\begin{eqn}
  \abs{\frac{\Vud \Vub^*}{\Vcd \Vcb^*}}
\end{eqn}

% oben rechts
\begin{eqn}
  \abs{\frac{\Vtd \Vtb^*}{\Vcd \Vcb^*}}
\end{eqn}

\begin{center}
  \begin{tikzpicture}[scale=6]
    \coordinate (A) at (0.25,0.70);
    \coordinate (B) at (1,0);
    \coordinate (C) at (0,0);
    \draw (A)--(B)--(C)--cycle;
    
    \tkzLabelSegment[above left=0.15](A,C){$\abs{\frac{\Vud \Vub^*}{\Vcd \Vcb^*}}$}
    \tkzLabelSegment[above right=0.15](B,A){$\abs{\frac{\Vtd \Vtb^*}{\Vcd \Vcb^*}}$}
    
    \tkzMarkAngle[size=0.35](C,A,B)
    \tkzLabelAngle[pos = 0.2](C,A,B){$\alpha$}
    
    \tkzMarkAngle[size=0.35](A,B,C)
    \tkzLabelAngle[pos = 0.2](A,B,C){$\beta$}

    \tkzMarkAngle[size=0.35](B,C,A)
    \tkzLabelAngle[pos = 0.2](B,C,A){$\gamma$}
  \end{tikzpicture}
\end{center}

% 2008: Nobelpreis für Kobayashi und Maskawa
% Unitaritätsdreiecke
% interessant: CP-Verletzung

\subsection{CP-Verletzung im Standardmodell}

% was ist direkte/indirekte CP-Verletzung?

% Entdeckung
\cite{wu}
\cite{cp-lee-yang}

% Kaon system
\cite{kaons-cronin-fitch}

% Was ist CP?
% Momentane Menge reicht nicht, um Materie-Antimaterie-Asymmetrie zu erklären
% Komplexe Phase in der CKM-Matrix -> CP-Verletzung
% CP-Dreieck (nicht rescaled, und rescaled)
% sin(2β) -> ermittelt über Amplitude der Mixing-Funktion

\subsection{$\PB$-$\PaB$-Oszillation}

% Warum interessant? -> sin(2β) lässt sich bestimmen, Test des Standardmodells

% Zeigen: es gibt Oszillation zwischen neutralen B-Mesonen (siehe Diagramme)
% Nehmen wir eine beliebige Überlagerung
% Hamiltonian sieht so aus:
Hamilton-Matrix
\begin{eqn}
  \begin{pmatrix}
    M_{11}-\frac{\I}{2} Γ_{11} & M_{12}-\frac{\I}{2} Γ_{12} \\
    M^*_{12}-\frac{\I}{2} Γ^*_{12} & M_{22}-\frac{\I}{2} Γ_{22} \\
  \end{pmatrix}
\end{eqn}
% Flavor-Eigenzustände sind nicht gleich den Massen-Zuständen: M_L, M_H

% Rechnung, die zeigt dass Übergangsamplitude wie 1 \pm cos(Δm_d) geht
% Berechnung der Mixing-Asymmetrie: (A - B) / (A + B)

\cite{babar-book}

\begin{figure}
  \begin{tikzpicture}[line width=1.2 pt, scale=1.1]
    \draw[fermion] (-1,2.3) node [left] {\Pqd} -- (0,2);
    \draw[fermion] (0,2) -- node [above] {$\Pqu,\Pqc,\Pqt$} (2, 2);
    \draw[fermion] (2,2) -- (3,2.3) node [right] {\Pqb};
     
    \draw[fermionbar] (-1,-.3) node [left] {\Pqb} -- (0,0);
    \draw[fermionbar] (0,0) -- node [above] {$\Pqu,\Pqc,\Pqt$} (2,0);
    \draw[fermionbar] (2,0) -- (3,-.3) node [right] {\Pqd};
    
    \draw[vector] (0,2) -- node [left] {\PW} (0,0);
    \draw[vector] (2,2) -- node [right] {\PW} (2,0);
  \end{tikzpicture}
  \hspace{1cm}
  \begin{tikzpicture}[line width=1.2 pt, scale=1.1]
    \draw[fermion] (-1,2.3) node [left] {\Pqd} -- (0,2);
    \draw[vector] (0,2) -- node [above] {\PWm} (2, 2);
    \draw[fermion] (2,2) -- (3,2.3) node [right] {$b$};
     
    \draw[fermionbar] (-1,-.3) node [left] {\Pqb} -- (0,0);
    \draw[vector] (0,0) -- node [above] {\PWp} (2,0);
    \draw[fermionbar] (2,0) -- (3,-.3) node [right] {\Pqd};
    
    \draw[fermion] (0,2) -- node [left] {$\Pqu,\Pqc,\Pqt$} (0,0);
    \draw[fermionbar] (2,2) -- node [right] {$\Pqu,\Pqc,\Pqt$} (2,0);
  \end{tikzpicture}

  \label{B_oscillation}
  \caption{bla}
\end{figure}


% B⁰ → J/ψ K*
\begin{tikzpicture}[line width=1.2 pt, scale=2]
  \draw[fermion] (0, 0.5) node [left] {\Pqd} -- (2, 0.5) node [right] {\Pqd};
  \draw[fermionbar] (0, 0) node [left] {\Pqb} -- (0.75, 0);
  \draw[fermionbar] (1.25, 0) -- (2, 0) node [right] {\Pqs};
  \draw[vector] (0.75, 0) -- node [above] {\PW} (1.25, 0);
  \draw[fermion] (1.25, 0) -- (2, -0.5) node [below] {\Pqc};
  \draw[fermionbar] (0.75, 0) -- (1.500, -0.5) node [below] {\Pqc};
\end{tikzpicture}

% wrong attempt at B⁰ → D- π+
%\begin{tikzpicture}[line width=1.2 pt, scale=2.0]
%  \node (create) at (1.25, 0) {};
%
%  \draw[fermion]  (0, 0.5) node [left] {\Pqd} -- (1, 0.5);
%  \draw[fermion]  (1, 0.5) -- (2, 1) node [right] {\Pqu};
%  \draw[fermionbar] (0, -0.5) node [left] {\Pqb} -- (1, -0.5);
%  \draw[fermionbar] (1, -0.5) -- (2, -1) node [right] {\Pqc};
%  \draw[vector]     (0.8, 0.5) -- node [left] {\PW} (0.8, -0.5);
%  \draw[fermionbar]    (create.center) -- (2, 0.375) node [right] {\Pqd};
%  \draw[fermion]       (create.center) -- (2, -0.375) node [right] {\Pqd};
%  \draw[gluon] (1, 0.5) -- (create.center);
%\end{tikzpicture}

%\begin{tikzpicture}[line width=1.2 pt, scale=1]
%  \draw[fermion] (-1,2.3) -- (0,2);
%  \draw[vector] (0,2) arc (160:0:1.025);
%  \draw[fermion] (2,2) -- (3,2.3);
%  %
%  \draw[fermion] (0,2) -- (1,1);
%  \draw[fermion] (1,1) -- (2,2);
%  %
%  \draw[vector] (1,1) -- (1,-.5);
%  \draw[fermion] (3,-1) -- (1,-.5);
%  \draw[fermion] (1,-.5) -- (-1,-1);
%\end{tikzpicture}

\section{Experimentelle Grundlagen}

% blabla

\subsection{Der LHCb-Detektor}

% Bild des Detektors
% 1, 2 Sätze zu den wichtigsten Komponenten

\subsection{Flavor-Tagging}

Zur Bestimmung der Oszillationsfrequenz $Δ m_{\Pqd}$ der produzierten $\PB$- und $\PaB$-Mesonen muss neben den Lebensdauern der einzelnen beobachteten Mesonen sowohl deren End-, als auch deren Anfangszustand bekannt sein.

Der Endzustand kann aus den Zerfallsprodukten der \PBz und \PaBz Mesonen bestimmt werden, wenn man einen Kanal mit self-tagging Endzustand wählt.
Zu diesen Kanälen gehören $\PBz \to \PDm \Pgpp$ und $\PBz \to \PJpsi \PKst$ (und die entsprechenden ladungskonjugierten Kanäle).

Die Bestimmung des Anfangszustandes, die man als \emph{Flavor-Tagging} bezeichnet, erweist sich als deutlich schwieriger.
Das LHCb-Experiment verwendet hierzu neuronale Netze, die darauf trainiert werden unter Einbeziehung weiterer während des Events erfolgter Beobachtungen eine möglichst gute Abschätzung für den Typ des \PB-Mesons (den sogenannten \emph{Tag}) zu liefern. % DaVinci erwähnen

Die verwendeten Tagger lassen sich in zwei Typen unterteilen: Opposite-Side-Tagger (OST) und Same-Side-Tagger(SST).

Die Opposite-Side-Tagger basieren darauf, dass \Pqb-Quarks fast ausschließlich in einem \Pqb\Paqb-Paar entstehen. Bildet dann z.B. das \Pqb-Quark ein \PaB-Meson, so könnte das ebenfalls entstandene \Paqb ein geladenes Meson, nämlich ein \PBp bilden, dessen Zerfallsprodukte einen Rückschluss auf seine Ladung und damit indirekt auf den Quarkinhalt des \PaB-Mesons ermöglichen.
So kann das geladene \PB-Meson z.B. semileptonisch ($\Pqb \to \Pqc \Plepton \Pnulepton$) zerfallen. Die Ladung des Leptons entspricht dann (bei Vernachlässigung von Prozessen höherer Ordnung) der gesuchten Mesonladung, die eine Bestimmung des \PB-Flavors erlaubt.\cite{ost}

Die Same-Side-Tagger basieren auf der Betrachung von Hadronisierungsprozessen bei der Entstehung des \PB-Mesons.
Ein entstandenes \Pqb-Quark benötigt zur Bildung eines \APB ein \APqd-Quark.
Dieses ist mit sehr hoher Wahrscheinlichkeit zusammen mit einem \Pqd entstanden, welches wiederum ein Hadron bildet, z.B. ein Pion oder Kaon.
Ist dieses geladen, so lässt sich aus der Ladung des Pions oder Kaons auf den Typ des \Pqd-Quarks im \PB-Meson schließen und damit sein Produktionszustand bestimmen.

Die so ermittelte Tag-Entscheidung ist nicht immer richtig; die Wahrscheinlichkeit, dass sie falsch ist bezeichnet man als Mistag-Wahrscheinlichkeit $ω$.
Für die Berechnung von $Δ m_{\Pqd}$ ist es in erster Linie nicht wichtig, $ω$ zu bestimmen.
Ein Fit der \PB-Oszillation ermöglicht aber nebenbei eine Bestimmung von $ω$ über die Amplitude der Schwingung.
%Demonstrieren
% D = 1 - 2ω

Zur Bewertung des Taggings lässt sich die Tagging-Effizienz $ε_\t{tag}$ einführen.
Diese ist einfach das Verhältnis zwischen der Anzahl getaggter und ungetaggter Signalereignisse.
Darüber lässt sich eine effektive Effizienz oder Tagging-Power als $ε_\t{eff} = ε_\t{tag} D^2$ definieren.
Diese macht eine Aussage über die statistische Genauigkeit des Tagging-Samples.

Wichtig ist eine genaue Kenntnis von $ω$ beispielsweise bei einer Messung der zeitabhängigen CP-Verletzung in der Oszillation neutraler \PB-Mesonen.
Hier misst man auf einem Kanal, der zwar nicht über einen self-tagging Endzustand verfügt, bei dem aber die Wahrscheinlichkeit, dass ein entstandenes Meson im Zustand \PBz oder \APBz zerfällt eine in der Zeit oszillierende Asymmetrie aufweist.
Die CP-Verletzung ist hier also für die Amplitude der beobachteten Schwingung verantwortlich.
Von kritischer Bedeutung für die Messung ist es daher, alle Faktoren, die die gemessene Amplitude beeinflussen, möglichst präzise zu bestimmen.
Dazu gehört neben der Zeitauflösung in erster Linie die Mistag-Wahrscheinlichkeit.

Da ein direkter Fit der Mistag-Wahrscheinlichkeit, anders als bei einem Kanal mit self-tagging Endzustand, hier nicht möglich ist, muss ein anderer Weg gefunden werden, um eine möglichst präzise Abschätzung von $ω$ zu erhalten.
Die verwendeten neuronalen Netze geben hierzu für jedes getaggte Event einen Mistag-Wert $η$ aus.
Dieser muss aber zuerst nicht $ω$ entsprechen.
Durch einen $\difference{m_{\Pqd}}$-Fit lässt sich aber eine direkte Gegenüberstellung des tatsächlichen (gefitteten) $ω$ und der Abschätzung $η$ erzeugen, die das Erstellen einer Parametrisierung erlaubt mit deren Hilfe man zu einem späteren Zeitpunkt $η$ zu einer korrigierten Abschätzung $η_C$ umrechnen kann.
Diesen Vorgang bezeichnet man als \emph{Tagging-Kalibration}.

\subsection{Das sPlot-Verfahren?}

\section{Datensatz}
\label{datensatz}

% Bd -> J/psi Kst 2012
% BDT

\section{\texorpdfstring{Kalibration des $SS\pi$-Taggers für $\PBzero \to \PJpsi \PKst$}{Kalibration des SSpi-Taggers für B0 -> JpsiKst}}

Das erste Ziel der vorliegenden Analyse ist es, eine Kalibration des Same-Side-Pion-Taggers auf dem in \ref{datensatz} beschriebenen Datensatz durchzuführen.
Das zweite Ziel ist ein Wiederholen der Analyse auf einer Version des beschriebenen Datensatzes, die mittels einer auf $\PBz \to \PDm\Pgpp$ gewonnenen Kalibrationsparametrisierung kalibriert wurde.

Zur Kalibration wird der Datensatz je nach Größe der Mistag-Vorhersage $η$ des neuronalen Netzes in fünf Kategorien aufgeteilt.
Das mittlere $η$ wird jeweils für alle Signalereignisse pro Kategorie ermittelt.
Die Trennung von Signal und Untergrund erfolgt hierbei über das sPlot-Verfahren\cite{splot}, wobei ausschließlich die Massenverteilung zur Unterscheidung von Signal und Untergrund verwendet wird.

Aus dem Datensatz bekannt sind die Massen- und Zerfallszeit-Verteilungen der rekonstruierten \PB-Mesonen, die Tag-Entscheidung des neuronalen Netzes sowie die Kenntnis über die Kaon-Ladung, die eine Identifizierung des Endzustandes erlaubt (\emph{self-tagging final state}).
Aus der Tag-Entscheidung und dem Endzustand lässt sich bestimmen, ob (bei korrektem Tag) ein Wechsel (Mixing) des \Pqb-Flavors stattgefunden hat.
Mit der Kenntnis des Mixings lässt sich ein Histogramm der Asymmetrie-Verteilung $M(t)$ aufstellen, welche in einem zweidimensionalen, simultanen Extended-Maximum-Likelihood-Fit über Daten aller fünf Kategorien gefittet wird, wobei die Zahl der Signal- und Untergrundereignisse sowie die Mistag-Wahrscheinlichkeit $ω$ individuell pro Kategorie gefittet werden.

Anschließend werden die 5 ermittelten $(η, ω)$-Wertepaare sowohl linear als auch quadratisch gefittet.

An dieser Stelle wird entschieden, welche der beiden Parametrisierungen für $η(ω)$ (linear, quadratisch) vorzuziehen ist.
Mittels der Koeffizienten $p_i$ aus dem Fit lässt sich eine Kalibrierung des Datensatzes durchführen, indem die $η$-Werte des neuronalen Netzes in kalibrierte Mistag-Werte $η_C$ mittels der Parametrisierung umgerechnet werden.

Interessant ist es, herauszufinden ob die Kalibrations-Parametrisierung eines Kanals angewendet auf einen anderen Kanal wiederum eine sinnvolle Kalibration ergibt.
Dies wäre ein Hinweis dafür, dass man die ermittelte Parametrisierung auch auf Zerfallskanäle anwenden könnte, bei denen ein Fit der Mistag-Wahrscheinlichkeit nicht möglich ist, zum Beispiel bei Kanälen, die CP-Verletzung zeigen. Ein wichtiges Beispiel ist $\PBz \to \PJpsi\PKshort$.

Ein mit auf $\PBz \to \PDm \Pgpp$ ermittelten Parametern kalibrierter $\PBz \to \PJpsi\PKsz$-Datensatz wurde bereitgestellt.
Die oben beschriebene Analyse wird auf dem kalibrierten Datensatz wiederholt.

\subsection{Parametrisierung der Likelihood-Funktion}

% Signal - Masse
\begin{eqns}
  P_\t{sig}(m; μ, σ_1, σ_2, σ_3) & = & f^{12}_{m;\t{sig}} G(m; μ, σ_1) + \\
  && (1 - f^{12}_{m;\t{sig}}) f^{23}_{m;\t{sig}} G(m;μ,σ_2) + \\
  &&(1 - (1 - f^{12}_{m;\t{sig}}) f^{23}_{m;\t{sig}}) G(m; μ, σ_3)
\end{eqns}

% Signal - Zeit
\begin{eqn}
  P_\t{sig}(t,q;τ,Δ m_{\Pqd}, ω, s_\t{sig} = M(t, q; τ, \Delta m_{\Pqd}, ω) \otimes R(t;s_\t{sig})
\end{eqn}

% m-t Fit in 5 Kategorien:
% Signal-pdf und zwei kombinatorische backgrounds
% jeweils Massen-pdf
% jeweils Zeit-pdf

% m-Fit für SPlot
% es werden die Massen-pdfs von oben verwendet

\begin{eqn}
  M_\t{sig}(t) = \frac{N_\t{unmixed}(t) - N_\t{mixed}(t)}{N_\t{unmixed}(t) + N_\t{mixed}(t)} = \cos(\difference{m_{\Pqd}} t)
\end{eqn}

\url{git@github.com:ibab/thesis}

\subsection{Implementierung und Durchführung des Fits}

\subsubsection{Implementierung des Wahrscheinlichkeitsmodells mittels des \texttt{RooFit}-Frameworks}
% wie genau mit RooFit implementiert?
% Startwerte nachlesbar hier und da

% RooGaussian, RooExponential, RooDecay, RooBMixDecay, RooSimPdfBuilder

\subsubsection{Durchführung der Optimierung}

\subsection{Fitresultate}

\begin{figure}
  \includegraphics[width=\textwidth]{analysis/JpsiKst-SSp/plot.pdf}
  \caption{Resultate der linearen und quadratischen Parametrisierungen der tatsächlichen Mistag-Wahrscheinlichkeiten $ω$ und der vom neuronalen Netz geschätzten Mistag-Wahrscheinlichkeiten $η$}
\end{figure}

% oder auch nicht {
% Alle Parameter (5 Kategorien) als Tabelle (Fehler mit Minos)
% Alle Parameter (Massen-Fit) als Tabelle (Fehler mit Minos)
% }

% nach Background-Trennung mit SPlot: Mittelwerte von η bestimmt

\subsection{Ermittlung der Kalibrationsparameter}

% wie gefittet?
% Tabelle der ermittelten ω, η
% Plot mit linearem Fit
% Entkorrelierung: Nullpunkt wird zum Schwerpunkt verschoben
% Tabelle mit p1, p2, (p3) + Korrelationsmatrix

\section{Schlussfolgerungen}

% vim: set ft=tex:
