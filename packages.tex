\usepackage{fixltx2e}
\usepackage{letltxmacro}
\usepackage{expl3}
\usepackage{xparse}
\usepackage{luacode}
\usepackage{luatexbase}
\RequireLuaModule{lualibs}
\usepackage{metalogo}

\usepackage{xcolor}
\usepackage{luacolor}

\usepackage[ngerman]{babel}
\usepackage[ngerman]{translator}
\usepackage[backend=biber,sortlocale=de_DE.UTF-8]{biblatex}
\usepackage[notbib,nottoc]{tocbibind}

\usepackage[sumlimits,intlimits,namelimits]{amsmath}
\usepackage{amssymb}
\usepackage{upref}
\usepackage{mathtools}
\usepackage[italic]{hepnicenames} % loads bm, before unicode-math
\usepackage[no-math]{fontspec} % after amssymb
\usepackage[math-style=ISO,bold-style=ISO,sans-style=italic,nabla=upright,partial=upright,vargreek-shape=unicode,warnings-off={mathtools-colon}]{unicode-math}
\usepackage[retainorgcmds]{IEEEtrantools}
\usepackage{tensor}
\usepackage[version=3]{mhchem}
\usepackage{siunitx}
\usepackage{esvect}

\usepackage{pdflscape}
\usepackage{float}
\usepackage[above,below,section]{placeins}
\usepackage{flafter}
\usepackage[margin=10pt,font=small,labelfont=bf]{caption}
\usepackage{graphicx}
\usepackage{array}
\usepackage{multirow}
\usepackage{booktabs}
\usepackage{subfig}

\usepackage[stable,bottom,hang]{footmisc}
\usepackage[strict]{csquotes}
\usepackage{hyphenat}
\usepackage{textcmds}
\usepackage{xspace}
\usepackage{eurosym}
%\usepackage{minted}

% Tikz %%%%%%%%%

\usepackage{tikz}
\usetikzlibrary{calc}
\usetikzlibrary{positioning}
\usetikzlibrary{shapes.geometric}
\usetikzlibrary{circuits.logic.IEC,circuits.ee.IEC}
\usetikzlibrary{arrows,shapes}
\usetikzlibrary{trees}
\usetikzlibrary{calc,through}
\usetikzlibrary{decorations.pathmorphing}	% For Feynman Diagrams
\usetikzlibrary{decorations.markings}

\pgfdeclaredecoration{complete sines}{initial}
{
    \state{initial}[
        width=+0pt,
        next state=upsine,
        persistent precomputation={\pgfmathsetmacro\matchinglength{
            \pgfdecoratedinputsegmentlength / int(\pgfdecoratedinputsegmentlength/\pgfdecorationsegmentlength)}
            \setlength{\pgfdecorationsegmentlength}{\matchinglength pt}
        }] {}
    \state{upsine}[width=\pgfdecorationsegmentlength,next state=downsine]{
        \pgfpathsine{\pgfpoint{0.50\pgfdecorationsegmentlength}{0.5\pgfdecorationsegmentamplitude}}
        \pgfpathcosine{\pgfpoint{0.50\pgfdecorationsegmentlength}{-0.5\pgfdecorationsegmentamplitude}}
    }
    \state{downsine}[width=\pgfdecorationsegmentlength,next state=upsine]{
        \pgfpathsine{\pgfpoint{0.50\pgfdecorationsegmentlength}{-0.5\pgfdecorationsegmentamplitude}}
        \pgfpathcosine{\pgfpoint{0.50\pgfdecorationsegmentlength}{0.5\pgfdecorationsegmentamplitude}}
}
    \state{final}{}
}

\tikzset{
	% >=stealth', %%  Uncomment for more conventional arrows
    vector/.style={decorate, decoration={complete sines,segment length=6}, draw},
	  provector/.style={decorate, decoration={snake,amplitude=2.5pt}, draw},
	  antivector/.style={decorate, decoration={snake,amplitude=-2.5pt}, draw},
    fermion/.style={draw=black, postaction={decorate},
        decoration={markings,mark=at position .55 with {\arrow[draw=black]{>}}}},
    fermionbar/.style={draw=black, postaction={decorate},
        decoration={markings,mark=at position .55 with {\arrow[draw=black]{<}}}},
    fermionnoarrow/.style={draw=black},
    gluon/.style={decorate, draw=black,
        decoration={coil,amplitude=4pt, segment length=5pt}},
    scalar/.style={dashed,draw=black, postaction={decorate},
        decoration={markings,mark=at position .55 with {\arrow[draw=black]{>}}}},
    scalarbar/.style={dashed,draw=black, postaction={decorate},
        decoration={markings,mark=at position .55 with {\arrow[draw=black]{<}}}},
    scalarnoarrow/.style={dashed,draw=black},
    electron/.style={draw=black, postaction={decorate},
        decoration={markings,mark=at position .55 with {\arrow[draw=black]{>}}}},
	bigvector/.style={decorate, decoration={snake,amplitude=4pt}, draw},
}
\tikzstyle{block} = [draw, rectangle, 
    minimum height=3em, minimum width=6em]


%%%%%%%%%%%%%%%%

\usepackage[unicode=true,pdfcreator={},pdfproducer={}]{hyperref}
%\usepackage{bookmark}
\usepackage[shortcuts]{extdash} % must be after hyperref for shortcuts
